\documentclass[10pt,a4paper]{article}
%\documentclass[12pt,a4paper]{article}

\usepackage[utf8]{inputenc}
\usepackage{hyperref}
\usepackage{polski}
\usepackage{graphicx}
\usepackage{listings}
\usepackage{algorithm}
\usepackage{algorithmic}

\usepackage{hyperref}
%\usepackage{antpolt}
\usepackage{amssymb}
\usepackage{multicol}
\usepackage{fancyvrb}

\usepackage{color}

%\setlength{\topskip}{0mm} \setlength{\footskip}{0mm} \setlength{\topmargin}{0mm} \setlength{\marginparwidth}{0mm}
%\setlength{\headsep}{2mm} \setlength{\headheight}{0mm} \setlength{\textheight}{250mm}
%\setlength{\textwidth}{160mm} \setlength{\oddsidemargin}{0mm} \setlength{\evensidemargin}{0mm}

\setlength{\topskip}{0mm} \setlength{\topmargin}{0mm}
\setlength{\oddsidemargin}{0mm} \setlength{\evensidemargin}{0mm}
\setlength{\marginparwidth}{0mm} \setlength{\headsep}{0mm}
\setlength{\headheight}{0mm} \setlength{\textheight}{240mm}
\setlength{\textwidth}{170mm}


\floatname{algorithm}{Algorytm}
\renewcommand{\lstlistlistingname}{Spis listingów}
\renewcommand{\lstlistingname}{Listing}


% \k{a} \'c \k{e} \l{} \'n \'o \'s
% \'z \.z \k{A} \'C \k{E} \L{} \'N
% \'O \'S \'Z \.Z 

\lstset{literate=
  {á}{{\'a}}1 {é}{{\'e}}1 {í}{{\'i}}1 {ó}{{\'o}}1 {ú}{{\'u}}1
  {Á}{{\'A}}1 {É}{{\'E}}1 {Í}{{\'I}}1 {Ó}{{\'O}}1 {Ú}{{\'U}}1
  {à}{{\`a}}1 {è}{{\`e}}1 {ì}{{\`i}}1 {ò}{{\`o}}1 {ù}{{\`u}}1
  {À}{{\`A}}1 {È}{{\'E}}1 {Ì}{{\`I}}1 {Ò}{{\`O}}1 {Ù}{{\`U}}1
  {ä}{{\"a}}1 {ë}{{\"e}}1 {ï}{{\"i}}1 {ö}{{\"o}}1 {ü}{{\"u}}1
  {Ä}{{\"A}}1 {Ë}{{\"E}}1 {Ï}{{\"I}}1 {Ö}{{\"O}}1 {Ü}{{\"U}}1
  {â}{{\^a}}1 {ê}{{\^e}}1 {î}{{\^i}}1 {ô}{{\^o}}1 {û}{{\^u}}1
  {Â}{{\^A}}1 {Ê}{{\^E}}1 {Î}{{\^I}}1 {Ô}{{\^O}}1 {Û}{{\^U}}1
  {Ã}{{\~A}}1 {ã}{{\~a}}1 {Õ}{{\~O}}1 {õ}{{\~o}}1
  {œ}{{\oe}}1 {Œ}{{\OE}}1 {æ}{{\ae}}1 {Æ}{{\AE}}1 {ß}{{\ss}}1
  {ű}{{\H{u}}}1 {Ű}{{\H{U}}}1 {ő}{{\H{o}}}1 {Ő}{{\H{O}}}1
  {ç}{{\c c}}1 {Ç}{{\c C}}1 {ø}{{\o}}1 {å}{{\r a}}1 {Å}{{\r A}}1
  {€}{{\euro}}1 {£}{{\pounds}}1 {«}{{\guillemotleft}}1
  {»}{{\guillemotright}}1 {ñ}{{\~n}}1 {Ñ}{{\~N}}1 {¿}{{?`}}1
  {ł}{\l{}}1 {ś}{{\'s}}1 {ź}{{\.z}}1  {ó}{{\'o}}1
}

\begin{document}
\pagestyle{empty}

%
%  strona tytułowa
%

\begin{center}
\textsc{\Huge{Uniwersytet Zielonogórski}}\\
\LARGE{Wydział Informatyki, Elektrotechniki i Automatyki}\\
\vspace{0.5cm}
\Large{Platforma .NET -- Projekt}\\
Prowadzący: dr inż. Marek Sawerwain\\ \vspace{1cm}
\LARGE{Tytuł raportu/sprawozdania}\\
\vspace{0.5cm} 
\Large{Wykonał: Damian Radecki, Grupa dziekańska: 33-INF-SSI-SP} \\
\Large{Projekt realizowano razem z: \\ Damian Kurkiewicz}\\
\Large{Data oddanie projektu: DD mmmm YYYY}
\vspace{1cm}
\begin{flushleft}
	Ocena: ..........................................
\end{flushleft}
\vspace{1cm}
\end{center}



%
% spis treści
%

\begin{multicols}{2}
	\footnotesize
	\tableofcontents
\end{multicols}


%
% spis listingów
%

\begin{multicols}{2}
	\footnotesize
	\lstlistoflistings
\end{multicols}




\noindent\makebox[\linewidth]{\rule{0.6\paperwidth}{0.4pt}}

\begin{flushleft}
	\emph{Motto:}\\
	\textit{Pisanie raportu przywilejem każdego studenta.}
\end{flushleft}

% 1. Wprowadzenie
% 2. Użyte technologie
% 3. Projekt i implementacji
% 4. Testy (jednostkowe, testy UI)
% 5. Wdrożenie -- instalacja
% 6. Wkład poszczególnych Autorek/Autorów projektu
% 7. Podsumowanie


\section{Wprowadzenie} 

\subsection{Aplikacje czasu rzeczywistego}
\hspace*{0.7cm} Aplikacje działające na żywo oferują wiele korzyści, które są sporym ułatwieniem dla użytkowników podczas używania takiej aplikacji. Czynności wykonywane
bez odświeżania strony nie tylko skracają czas wykonywania czynności czy obsługiwania samej witryny to jeszcze znacznie ułatwiają komunikację, unikają 
blokowania strony i tworzą bardziej intuicyjny interface. Takie aplikacje internetowe staja się normą w dzisiejszych czasach. Każde przeładowanie strony jest
nie komfortowe i stwarza pewnego rodzaju niebezpieczeństwo wykradnięcia danych. Serwisy internetowe obsługujące komunikację real-time z klientem są lepiej 
zabezpieczone i działają wydajnościowo lepiej. Powstawało wiele technologii do wsparcia komunikacji na żywo, które działają zarówno po stronie witryny i
serwera. Są to między innymi WebSocket, SignalR, RabbitMQ czy Apache Kafka. Wszystkie z nich są dziś globalnie używane do wsparcia przekazu informacji.

\subsection{Opis działania}
\hspace*{0.7cm} Projekt chatu na żywo jest aplikacją, która wspiera komunikację między użytkownikami, aby ich konwersacje nie działały w stylu w jakim działa klasyczny serwer
e-mail. Założenie projektu są takie, aby użytkownicy bez przeładowania strony mogli wymieniać miedzy sobą wiadomości. Dodatkowo wszelkie powiadomienia
przychodzą również bez zbędnego odświeżania witryny. Sprawia to, że witryna jest bardziej intuicyjna, łatwiejsza i szybsza w obsłudze. Taka architektura
aplikacji jest przyjazna użytkownikowi, od którego będzie wymagana minimalny wysiłek w trakcie używania strony. Celem takiej aplikacji jest też maksymalne 
bezpieczeństwo wspierane przez bearen token i autoryzację użytkowników z zachowaniem szyfrowania danych poufnych. Architektura zapewnia, że nie będziemy otrzymywać wiadomości od niezaakceptowanych użytkowników lecz daje możliwość uczestnictwa w czatach posiadających osoby nieznajome poprzez mechanizm grup. W grupach każdy użytkownik może zaprosić swoich znajomych co może spowodować komunikację między nieznajomymi w danym czacie.


\subsection{Grupa docelowa}
\hspace*{0.7cm} Grupą docelową są wszyscy użytkownicy którzy cenią sobie bezpieczeństwo i wygodę. Chcą szybko skomunikować się ze swoimi przyjaciółmi bądź grupą docelową
bez żadnych opóźnień czy niepotrzebnych przeładować strony. Są pewni tego, że ich dane są przechowywane w bezpiecznym miejscu i nikt nie wkradnie się na ich
konto. Mogą to być zarówno firmy, które chcą komunikować się między sobą i ewentualnie z klientami poprzez utworzenie chatu dla grupy użytkowników jak i dla szkół, uniwersytetów, grup pracowników, przyjaciół czy kolegów. 


\section{Użyte technologie}

\subsection{.Net Core} 			%razem
\hspace*{0.7cm} Popularny, nowoczesny i wydajny framework oparty o otwartoźródłowa implementację, który został wydany w 2016 roku do ogólnego przeznaczenia. Stanowi zestaw bibliotek pozwalający tworzyć wieloplatformowe aplikacje o wysokim stopniu bezpieczeństwa. Framework ten pozwala na pisanie aplikacji przeznaczonych do obliczeń chmurowych, IoT oraz jak w naszym przypadku do pisania web serwisu. 

\begin{figure}[h]
	\centering
	\includegraphics[width=0.3\linewidth]{dotnet5_platform}
	\caption{Logo .Net Core}
	\label{fig:dotnet5platform}
\end{figure}

Framework .Net Core został przez nas wybrany, ponieważ jest to nowa oraz dobrze prosperująca technologia wprowadzająca dużą dawkę świeżości podczas tworzenia nowego oprogramowania. Posiada wsparcie dla tworzenia web serwisów opartych o metodykę REST, poprzez dodanie nowych i gotowych do działania bibliotek. Platforma .Net Core jest znacznie wydajniejsza od .Net Framework. Wprowadza znaczące usprawnienia przekładające się na szybkość działania pisanych programów.

\subsection{Entity Framework}   %ja
\subsection{SignalR} 			%kurek
\subsection{MySql} 				%ja
\subsection{Angular} 			%kurek

\section{Projekt}

\subsection{Struktura projektu}			%razem
\subsection{Use Cases}					%razem
\subsection{Struktura bazy danych}  	%ja
\subsection{Komuniakcja z web servicem} %kurek

\section{Implementacja}

\subsection{Diagram klas}					%razem - chyba
\subsection{Modele bazy danych}				%ja - enity framework
\subsection{Modele DTO} 					%kurek
\subsection{Implementacja kontrolerów}		%razem
\subsection{Problemy i ich rozwiązania}		%razem

\section{Testy}

\subsection{Testy jednostkowe}   %ja
\subsection{Testy integracyjne}  %kurek

\section{Opis wkładu własnego w realizację projektu}

%ja
\subsection{Stworzenie i konfiguracja bazy danych}
\subsection{Stworzenie systemu logowania i rejestracji}
\subsection{Autoryzacja i autentykacja użytkowników}
\subsection{Konfiguracja środowiska i serwera}

%kurek
\subsection{Strona internetowa opracowana w Angular}
\subsection{Implementacja zarządzania wiadomościami}
\subsection{Implementacja zarządzania powiadomieniami}
\subsection{Implementacja zarządzania chatami}
\subsection{Obsługa SignalR}

\section{Podsumowanie}

\subsection{Wnioski}
\subsection{Do zrealizowania przy dalszym rozwoju}


\end{document}
